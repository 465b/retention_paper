%% Copernicus Publications Manuscript Preparation Template for LaTeX Submissions
%% ---------------------------------
%% This template should be used for copernicus.cls
%% The class file and some style files are bundled in the Copernicus Latex Package, which can be downloaded from the different journal webpages.
%% For further assistance please contact Copernicus Publications at: production@copernicus.org
%% https://publications.copernicus.org/for_authors/manuscript_preparation.html


%% Please use the following documentclass and journal abbreviations for preprints and final revised papers.

%% 2-column papers and preprints
\documentclass[npg, manuscript]{copernicus}


%% \usepackage commands included in the copernicus.cls:
%\usepackage[german, english]{babel}
%\usepackage{tabularx}
%\usepackage{cancel}
%\usepackage{multirow}
%\usepackage{supertabular}
%\usepackage{algorithmic}
%\usepackage{algorithm}
%\usepackage{amsthm}
%\usepackage{float}
%\usepackage{subfig}
%\usepackage{rotating}
\usepackage{hyperref}


\begin{document}

\title{Phytoplankton Retention Mechanisms in Estuaries: A Case Study of the Elbe Estuary}


% \Author[affil]{given_name}{surname}

\Author[1]{Laurin}{Steidle}
\Author[2]{Ross}{Vennell}

\affil[1]{Universität Hamburg, Olbersweg 24, 22767 Hamburg, Germany}
\affil[2]{Cawthron Institute, 98 Halifax Street East,
Nelson 7010, New Zealand}

%% The [] brackets identify the author with the corresponding affiliation. 1, 2, 3, etc. should be inserted.

%% If an author is deceased, please mark the respective author name(s) with a dagger, e.g. "\Author[2,$\dag$]{Anton}{Smith}", and add a further "\affil[$\dag$]{deceased, 1 July 2019}".

%% If authors contributed equally, please mark the respective author names with an asterisk, e.g. "\Author[2,*]{Anton}{Smith}" and "\Author[3,*]{Bradley}{Miller}" and add a further affiliation: "\affil[*]{These authors contributed equally to this work.}".


\correspondence{Laurin Steidle (laurin.steidle@uni-hamburg.de)}

\runningtitle{TEXT}

\runningauthor{TEXT}





\received{}
\pubdiscuss{} %% only important for two-stage journals
\revised{}
\accepted{}
\published{}

%% These dates will be inserted by Copernicus Publications during the typesetting process.


\firstpage{1}

\maketitle



\begin{abstract}
    Due to their role as primary producers, phytoplankton is essential for the proliferation of estuarine ecosystems.
    However, as small drifting organisms, phytoplankton cannot resist currents.
    Estuaries not only represent challenging habitats due to a strong salinity gradient, but also have net transport into oceans.
    So far, little is known about how phytoplankton retain themselves in estuaries.
    We present a new individual-based Lagrangian model of the Elbe estuary examining possible retention mechanisms of phytoplankton. In more detail, we investigated how reproduction, sinking and rising, as well as diel vertical migration might allow for population survival. 
    We find that rising favors the retention success, fast sinking cause the opposite.
    We further provide first estimates on outwashing losses.
    Our simulations illustrate that shallow stretches of water along the river banks are essential for the long-term survival of phytoplankton populations.
    These results give important indications for ecosystem-based management of estuaries.
\end{abstract}


% \copyrightstatement{TEXT} %% This section is optional and can be used for copyright transfers.


% \introduction  %% \introduction[modified heading if necessary]
\section{Introduction}

Estuaries represent highly productive ecosystems in relation to their area and contribute to the global carbon cycle along with their role as a source of nutrients and hatching grounds for marine ecosystems \citep{Cloern2014,Arevalo2023}.
% (Phytoplankton primary production in the world’s estuarine-coastal ecosystems,Fish larvae dynamics in temperate estuaries: A review on processes, patterns and factors that determine recruitment). 
While they are heavily influenced by anthropogenic stressors such as diking, dredging, and fishing which can be seen as a stressor on the ecosystem, they are of tremendous importance for anthropogenic usage 
\citep{Jennerjahn2013, Brown2022, Wilson2002}.
Estuaries present challenging, dynamic  of its residents due to the strong salinity gradient and net transport to the sea.
Here, we explore how phytoplankton, drifting small primary producers that form the basis of estuarine food webs, can sustain themselves in such dynamic environments.
% (Pressures, stresses, shocks and trends in estuarine ecosystems: An introduction and synthesis,
% Anthropogenic-estuarine interactions cause disproportionate greenhouse gas production: A review of the evidence base,
% Productivity, Fisheries and Aquaculture in Temperate Estuaries)

As for most ecosystems - their dynamics are strongly controlled by primary producers \citep{Chen2023}.
% (Advances in phytoplankton population ecology in the Pearl river estuary).
Phytoplankton represent the base of aquatic ecosystems. 
Apart from biofilm-forming phytoplankton which are attached to their substrate \citep{Cheah2022},
%A methodological review on the characterization of microalgal biofilm and its  extracellular polymeric substances 
the vast majority of phytoplankton organisms is drifting passively and cannot withstand currents.
% The exception being biofilm-forming or benthic phytoplankton which are attached to their substrate (cite).
With the estuary having a net outwards flow, we would expect phytoplankton to be moving downstream over time and to be washed out from limnic waters, via brackish to marine waters.
Hence, the question arises how phytoplankton as the drifting base of estuarine food webs is able to maintain their population size without declining due to the transport into the open ocean.
If we assume that the population is not exclusively maintained by upstream phytoplankton that is washed into the estuary, then there must be some sort of retention mechanism that enables a phytoplankton population to remain in the estuary.
% While we suspect that such a mechanism is crucial for survival, it is currently not well understood due to the difficulty in either measuring or modeling such processes.



% \subsection{Current state of knowledge}

So far, different theories exist on how estuarine phytoplankton is able to maintain its position. Previous observational studies suggested several possible mechanisms that can allow for the retention of phytoplankton populations in estuarine systems - vertical migration in the form of sinking,  rising or diel migration, stickiness.

% \texttt{Comment on sinking and rising?}
Diel vertical migration is a process where organisms move up and down in the water column in dependence of the sun phase.
This movement can be favorable for retention as it allows particles to reduce the time in the faster downstream currents at the water surface.
However, this has only been demonstrated for larger motile estuarine organisms such as zooplankton \citep{Hall2015, Kimmerer2002, Crawford1991,Hall2011}.
% (Vertical spatio-temporal patterns of phytoplankton due to migration behaviors in two shallow, microtidal estuaries: Influence on phytoplankton function and structure,
% Persistence of Tidally-oriented Vertical Migration by Zooplankton in a Temperate Estuary,
% Evidence for avoidance of flushing from an estuary by a planktonic, phototrophic ciliate, Flocculation kinetics and mechanisms of microalgae- and clay-containing suspensions in different microalgal growth phases).
Yet, a study by \citep{Anderson1985} showed that diel migrating dinoflagellates were able to out compete other non-motile phytoplankton in an embayment environment and even compensate their outwashing losses through reproduction increasing their abundance.
However, this also implies that the growing part of the population is somehow retaining their position.
If the regrowing population is also continuously drifting downstream they will not able to sustain their population in that area and ultimately die out due to unfavorable salinity conditions in marine waters \citep{Admiraal1976, vonAlvensleben2016, Jiang2020}.

Estuaries are complex and strongly dynamic systems such that it is still difficult to predict ecosystem dynamics or the effects of anthropogenic due to their complex bathymetry \citep{Michael2016, Fringer2019}. 
% (Sensitivity analysis of three-dimensional salinity simulations in North San Francisco Bay using the unstructured-grid SUNTANS model,
% The future of coastal and estuarine modeling: Findings from a workshop)
Nevertheless, previous modelling studies have investigated potential retention mechanisms of phytoplankton in estuaries.
In \citep{Simons2006,Kimmerer2014} they use a particle tracking model also known as Lagrangian model to study Zooplankton retention.
\citep{Simons2006} examined the dispersal and flush time of mussel larvae in the St. Lawrence Estuary while \citep{Kimmerer2014} examined zooplankton movement in the San Francisco Estuary.
They were able to show that sinking and diel vertical migration slows the outwashing process and might be a beneficial retention tactic.
However, they did so by ignoring key processes like reproduction, mortality, and stranding or sedimentation processes.
Moreover, both studies were based on low resolution structured grid models, which we suspect to under-represent the complex bathymetry of estuarine systems \citep{Ye2018}.

Diatoms or benthic microalgae in particular have been observed to be strongly negatively buoyant, hence sinking to the ground and remaining there for a long time \citep{Passow1991,ThomasAnderson1998}
% (Species-specific sedimentation and sinking velocities of diatoms).
Studies also found sticky compounds in phytoplankton aggregates that are suspected to let them stick to suspended particles making them sink to the ground or sticking to their surroundings aiding retention \citep{Kiørboe1993,VanderLee2000}.
% \texttt{Shall I remove sticking from intro as it is not tested in my model?}

In summary, different retention mechanisms have been observed or examined in modeling studies.
However, they did so in either in isolation in the case of the observational studies or with major simplifications for the modeling studies.
There is currently a lack of theoretical studies that allow a more comprehensive overview of possible retention mechanisms. 


Here, we are exploring possible retention mechanisms of phytoplankton using the Elbe estuary as a case study.
It is located in the north of Germany and flows into the North Sea.
Like most alluvial estuaries, it is relatively shallow, with most of it averaging only a few meters in average depth.
Similar to other European estuaries it experienced a strong anthropogenic pressure over the last centuries.
Most notably diking to restrain it to a narrow channel and dredging to improve access to the Hamburg harbor.
Unlike other major European harbors, the Hamburg harbor is located far inwards roughly 100 \unit{km} away from the coast.
This causes the main channel to experience a sudden jump in bathymetry from approx. 5 \unit{m} at border of the city to approx. 20 \unit{m} in the harbor (see fig. \ref{fig:bathymetry}). 
This bathymetric jump is suspected to be the cause of a collapse of the phytoplankton biomass causing an increase in oxygen depletion and high ammonium remineralization after the bathymetric jump \citep{Schroeder1997,Holzwarth2018,Sanders2018}.
Ongoing dredging is being carried out to maintain the depth of the navigational channel causing high turbidity \citep{Kappenberg2001}.
While important aspects of the along-channel biochemical dynamics have been studied, little is known about their vertical and shore-to-shore dynamics \citep{Goosen1999, Dahnke2008, Sanders2018}.

\begin{figure}
    \includegraphics[width=8.3cm]{figures/harbor.png}
    \caption{Bathymetry of the Elbe model around Hamburg. Note the bathymetric jumps from ~5 \unit{m} on the right, upstream side to a short 10 \unit{m} step in the upper harbor area to the ~20 \unit{m} in the lower harbor area all the way to the North Sea. Also note that there exists no channel that does not pass through the 10 \unit{km} of exclusively 20 \unit{m} deep channel.}
    \label{fig:bathymetry}
\end{figure}

For this purpose, we further developed the individual-based Lagrangian model OceanTracker \citep{Vennell2021} and applied it to the Elbe estuary using the hydrodynamics calculated by a recent model SCHISM \citep{Pein2021}.
While the Lagrangian model simulated the movement of the inanimate organism, we included key phytoplankton features such as reproduction and mortality, sinking and rising, as well as diel vertical migration. % \textit{and dormancy?}
Using this model, we investigate under which conditions phytoplankton retention can be reproduced.

\section{Methods}

\subsection{Model description}

%\subsubsection{Lagrangian model}

In our study we use a Lagrangian approach with the particle tracking model OceanTracker \citep{Vennell2021}.
While particle tracking on unstructured grids has been relatively computationally expensive until recently \citep{Vennell2021}, it offers several advantages.
Firstly, it allows us to reuse already computed hydrodynamic models to model tracer-like objects.
This is overall much faster than recalculating the advection-diffusion-equation in an Eularian model.
Secondly, because we are simulating individually particles we are able to observe their tracks.
This makes the interpretation of our results more intuitive and allows us to include individual based properties and processes that can only be parametrized in Eulerian models.

%\subsubsection{Hydrodynamics data}
We use the hydrodynamics data of the most recent SCHISM model of the Elbe \citep{Pein2021}.
This model uses a three-dimensional unstructured  grid to represent the full Elbe estuary from the weir at Geesthacht to the North Sea, including several side-channels and the harbor area (see fig. \ref{fig:model_domain}). 
% (Look for more details in pein paper that might be interesting)
The model provides us with a node-based mesh containing a range of information such as water velocity, salinity, water level and dispersion.
The year represented in that dataset is 2012 with a temporal resolution of 1 hour and a dynamically varying spacial resolution with a median distance between nodes of approx. 60 \unit{m}.

\begin{figure}
    \includegraphics[width=12cm]{figures/dfg_foto_wettbewerbg_laurin_legendless_steidle_small.png}
    \caption{Map of the full model domain, with Geesthacht being the upstream boarder on the right and the North-sea being the downstream border on the left. The gray outline marks the edge of the model domain. Blue and green dots show an example snapshot of a phytoplankton distribution in the model. Blue represents floating, green settled particles. The background map has been provided by OpenStreetMap.}
    \label{fig:model_domain}
\end{figure}

%\subsubsection{Particle based processes}

We add a set of biological features on top of the otherwise inanimate organism.
These features include reproduction and mortality, vertical movement in form of sinking, rising or diel vertical migration, stranding, and sedimentation.

Reproduction is represented as a chance of creating copies of themselves. 
We perform several simulations for a range of reproduction rates that are constant over the lifetime of the particle and the chance to reproduce is evaluated every minute.

Mortality is induced either by high salinity, when they dry-out while stranded, or due to long term light limitation.
When particles are exposed to high salinity water above $20$PSU, a mortality probability of $1\%$ per minute is imposed.
This threshold is chosen based on the salinity tolerances presented in \citep{vonAlvensleben2016}.
This is only an approximation and many phytoplankton species deviate from this number.
However, the main motivation for this choice is that most of the particles that die through this process have passed the isohaline for several hours and are assumed not to return again through this isohaline.
We consider particles that are stranded outside the water and lie dry for more than $7$ consecutive days to be dead and remove them.
Particles are also expected to be dead and are removed if they are light-limited for $28$ days \citep{Walter2017}.
They are considered light-limited at a depth of $1 \unit{m}$ based on turbidity data presented in \citep{Pein2021}.
A particle starts its life with a light budget of $28$ days, and each minute below $1 \unit{m}$ reduces this budget by $1$ minute, while the opposite applies when they are above $1 \unit{m}$.
Children of light-limited parents inherit the remaining light budget of their parents.

\medskip

We study the effect of different patterns of vertical motion.
The first is monodirectional upward or downward vertical motion, representing either positively or negatively buoyant phytoplankton organisms.
This buoyancy can be interpreted either as an active choice of buoyancy by the organism through adaptation, or as governed by the aggregate in which the organism is embedded.
For monodirectional vertical motion, we assign each particle a vertical velocity, which remains constant throughout its lifetime.
The second mode of vertical motion is diel vertical migration.
Here the particles change their direction of motion based on the current phase of the sun, creating a motion pattern where they rise during the day and sink during the night using the same velocity range.

% A large part of the Estuary is flat and regularly falls dry, making stranding and resuspension process important to predict their trajectories. 
We include a sedimentation and resuspension model to represent tidal stranding and particles settling on the bed of the estuary.
Particles become stranded when the current grid cell falls dry. 
They are not allowed to disperse into dry cells.
A grid cell is considered dry if at least one of the nodes of the cells' triangle is considered dry by the SCHISM wet flag.
Once this cell is rewettet all stranded particles resuspend.
Particles settle on the bed once they attempt to move below the bottom model boundary and are resuspended based on a critical sheer velocity of 0.009 \unit{m s^{-1}} (see appendix for details).
The velocity profile in the bottom layer, or log layer, is calculated by
% centered equation
\begin{equation}
    U(z) = \frac{u_{*}}{\kappa} ln \frac{z}{z_{0}},
\end{equation}
where U is the friction velocity representing the drag at height z  above the seabed, $\kappa$ is the van Karman constant, $z_{0}$ is a length
scale reflecting the bottom roughness, and $u^{*}$ is the critical friction velocity.
If the friction velocity is above the critical friction velocity the particle is resuspended.
Particles that are stranded or settled on the bed are allowed to reproduce.
Particles are not only advected but also dispersed based on eddy diffusivity.
This allows us implement a dynamic dispersion that is crucial to represent tidal-pumping processes.
Dispersion was modeled using a random walk using a random number generator with a normal distribution.
Horizontally the standard distribution of the random walk was set to 0.1 \unit{m s^{-1}}.
The displacement by vertical dispersion $\partial z$ of particle $i$ is calculated by
\begin{equation}
    \partial z_{i} = K_{v}^{'}(z_{i}(n))\partial t + N(0, 2K_{v}(z_{i}))
\end{equation}
based on \citep{Yamazaki2014} where $z_{i}$ is the vertical position of the particle, $K_{v}^{'}$ is the vertical eddy diffusivity gradient, $K_{v}$ is the vertical eddy diffusivity and $N$ is the normal distribution.
The term based $K_{v}^{'}$ is needed to avoid particle accumulation on the top and bottom of the water column.

For each particle we log their distance traveled, age, water depth, and status (whether they are drifting or settled on the river bank or bottom). This allows us for example to compare successfully retained particles (older than three months) unsuccessfully retained particles (dead after less than three months).
These observables are measured every 12 hours starting at midnight.

%\subsubsection{Numerics}
Model simulations and visualizations were performed in Python making heavy use of Numba, a LLVM-based Python JIT compiler \citep{Lam2015} to drastically speed up the simulations \citep{Vennell2021}.
Trajectories were calculated using a second order Runge-Kutta scheme with a fixed time step of 10 \unit{s}.
Flow velocities, like any other hydrodynamic data, were interpolated linearly in time and space using barycentric coordinates.

\subsection{Experimental configurations}

We perform two sets of experiments to test the influence of different vertical movements on the retention success of phytoplankton in the Elbe estuary.

In the first experiment, we examine a range of different monodirectional upward or downward particle velocities from $-10$ to $+10$ \unit{mm s^{-1}} in $2$ \unit{mm s^{-1}} steps representing sinking or rising phytoplankton organisms \citep{Fennessy1996}.
Each vertical velocity is examined for a range of different growth rates with effective population doubling times ranging from 772 to 77 days with a logarithmic scaling.
Hence, a total of 187 different scenarios are tested.
In the second set of model experiments, we study the influence of possible diel vertical migration patterns for the same vertical velocities and growth rates.

In both set of experiments, we release the studied phytoplankton population at the beginning of the year homogeneously distributed in a volume covering the full water column at the weir in Geesthacht and examine how the population distributes itself over the estuary and whether it is able to maintain its population size over time.
Conceptually, we consider a population to be successfully retained if it is showing long term growth. 
Practically, this is evaluated by comparing the population size at the end of the year to the size after release.
The first three months of the simulations are considered as initial model spin-up time during which the initial population is dispersed downstream throughout the estuary.
Population size changes are measured at the end of the year relative to the population size after this initial spin-up time.

\section{Results}

\subsection{Retentions success in different scenarios}

\begin{figure}
    \includegraphics[width=8.3cm]{figures/retention_success_combined.png}
    \caption[]{
        Relative population changes for the monodirectional movement (a) and diel migration (b) scenarios. Positive vertical velocities indicate an upwards drift. Positive population changes represent a retention success (green) while negative population changes represent a loss of the population (brown).
        \texttt{I think I'll redo the simulation and shift the pop doubling time slightly to the left}
        }
    \label{fig:retention_success}
\end{figure}

The results of the retention experiments are visualized as heatmap in fig. \ref{fig:retention_success}.
Fig. \ref{fig:retention_success}a shows the results for the monodirectional vertical migration scenarios i.e. constant sinking or rising. Fig. \ref{fig:retention_success}b shows the results for the diel vertical migration scenarios. Each pixel in the heatmap represents a simulation with a specific combination of vertical velocity and growth rate. The coloring indicates the relative population change after one year. White pixels and the boundary between green and brown pixels represent net-zero growth rate simulations. In this case, the losses are equal to the growth. Therefore, we can use the growth rate as an estimate for the total relative losses due to downstream transport, drying out while being stranded, and light starvation.

Our simulations show that the population is able to successfully retains itself under certain conditions. For the case of the monodirectional movement we see that a higher positive velocity (representing buoyancy) and a higher growth rate is more beneficial for retention success than a downward oriented velocity (sinking) and lower growth rates.
As expected, simulations in which the reproduction is set to zero do not show any retention success.
While it is easy to understand that high growth rates aid retention, we were surprised that buoyant phytoplankton particles are more successful in maintaining their growth in an estuary than sinking ones.

For the case of the diel vertical migration we see a equal or higher retention success compared to no vertical velocity with velocities in the range of 4-10 \unit{mm s^{-1}} vertical velocities but shows to me less successful than the monodirectional movement for all cases.
A vertical velocity of 2 \unit{mm s^{-1}} is less successful then no migration.
Most importantly, none of the diel migration scenarios are improving the retention success compared to passively drifting organisms.


\subsection{Spatial analysis}

We are now taking a closer look at spacial factors that allow phytoplankton particles to maintain net growth in the estuary. 
For this analysis we used data from both sets of experiments.
Fig. \ref{fig:migration-long-vs-short} compares two box plots showing the average water depth at the location of each particle between particles that remained alive for less than three months (short-living) and for more than three months (long-living).
Depth is measured relative to the current water surface.
Hence, a value above zero indicate that the particle is stranded on the river bank.
These analyses demonstrate that long-living particles predominantly live close to the river banks in shallower waters or on tidal flats.

\begin{figure}
    \includegraphics[width=8.3cm]{figures/retention_boxplot.png}
    \caption[]{Box and violin plot showing the vertical distribution of particles.  Short-living are those younger than 3 months and long-living all those older than that. Depth is measure relative to the current water surface with positive numbers being above the water surface i.e. stranded on shore.}
    \label{fig:migration-long-vs-short}
\end{figure}

We moreover analyzed the horizontal spacial distribution of long and short-living particles in fig. \ref{fig:migration-long-vs-short-heatmap}.
To do this, we divide the model domain into equally sized hexagons.
The color of each hexagon indicates the average age of the particles within it.
Hexagons with a yellow color indicate an average age of over three months.
These yellow areas are mainly found along the river banks in shallow waters or tidal flats.

\begin{figure}
    \includegraphics[width=8.3cm]{figures/age_hexbin.png}
    \caption[]{
        Hex-bin heatmap of the Elbe with Hamburg in the bottom right showing average particle age per bin.
        Colors indicate the age of the particles, with yellowish colors indicating an average age of over three months.
        Yellow areas are mainly found along the river banks in shallow waters or tidal flats.
        The largest areas are the Mühlenberger Loch (a) and the Haseldorfer Binnenelbe (b).
        Smaller areas can also be found at Wedeler Marsch (c), at the mouth of Wischhafener Süderelbe (d), Ruthenstrom, and Stör (e), and at Nordkedding (g) and Neufelder Marsch (h).
        \texttt{I am aware that the unit of age (days) is still missing. This figure will be modified slightly before publication.}
        }
    \label{fig:migration-long-vs-short-heatmap}
\end{figure}

To further investigate the reasons for the positive effect of buoyancy and the importance of shallow waters and tidal flats, we repeated the first set of simulations and disabled the reproduction of settled and stranded particles.
Under these conditions, populations were unable to retain themselves in the estuary, regardless of their vertical velocity and growth rate indicating that tidal flats are essential for the survival of the population.


%\subsubsection{Discussion}


\subsection{Interpretation and contextualization of Results}

In this study, we investigated different strategies to explain how phytoplankton populations are able to maintain their growth in estuaries while constantly being at risk to be transported into the open ocean.
% We find that reproduction rates in the range of XX-XX are sufficient for phytoplankton organisms to survive in the estuary under certain conditions.
% In addition, we found that buoyancy aids the retention success, whereas sinking decreases the retention success. Finally, especially shallow areas near the river banks and tidal flats represent important habitats for phytoplankton organisms and allow them to survive, whereas deeper areas lead to a faster transport of estuarine phytoplankton into the marine waters.

The limit for population doubling times that we found to be necessary for survival are of the order of around 200 days. Doubling times typically realized in natures are of the order of a few days which is two magnitudes small then those that we found necessary in our model \citep{Koch2004,Wirtz2011}.
The low growth rates required for successful retention demonstrate that our model is also meaningful under more realistic environmental conditions, for example if maximum growth rates cannot be reached due to nutrient or temperature limitation.

Our results suggest that shallow areas are very important for maintaining the estuary phytoplankton population.
Plankton that lives in this area falls dry on the shore on a regular basis due to the tides. 
Therefore, they do not move for much of the day.
We further see that positively buoyant plankton is more successful in retaining themselves. 
This is probably because they are more likely to be transported high up on the river bank where the water is less likely to reach them.
This effect is emphasized in flatter regions as the distance between the wash margin and constantly flooded areas is larger increasing the chance of settlement or them stranding again.


Initially, we expected sinking particles to have a higher retention success than buoyant ones. 
However, we found that faster sinking particles are less successful in retaining themselves.
Sinking velocities of less then $2 \unit{mm s^{-1}}$ are common for diatoms \citep{Passow1991} while larger velocities have been observed for aggregates in the Elbe estuary \citep{Fennessy1996}.
Sinking particles have a reduced downstream velocity because they sink to the river bed, where they move more slowly, or are stationary if they are temporarily settled or stranded.
In addition, the deeper water layer of the Elbe typically has a higher upstream velocity than the upper water column due to temperature-induced density stratification \citep{Pein2021}.
Nevertheless, buoyant particles were more successful in their retention in our simulations.
The low chance of survival in the estuary for sinking particles might be explained by their light limitation in deeper waters.
We expected particles to die if they are exposed to dark conditions for more than 2 weeks.
Thus, sinking particles have a disadvantage to buoyant particles since they are more likely to become light limited and eventually die.
This suggests that dredging has a negative impact on sinking plankton because it increases both depth and turbidity \citep{DeJonge2014}, which increases the aphotic depth and therefore the volume of dark water relative to the amount of light water.

We suspect that the reason for the increased retention success of diel migrating organisms is similar to the monodirectional case.
When the upwards diel migration coincides with high tide, particles are more likely to be stranded far out on the shore, reducing their risk of being washed out quickly.
The higher the upward velocities, the greater the chance of being in the farthest part of the water body during high tide.
However, because they are sinking for half of the day they also tend to be light limited more frequent than positively buoyant particles.
It seems like these favorable and unfavorable processes balance each other out, resulting in a similar retention success as for the monodirectional case.


\subsection{Model limitations \& future perspectives}

In this study, we aimed to thoroughly investigate different possible retention mechanisms in a complex Lagrangian model system with a highly resolved bathymetry.  
Due to this computational and spatial complexity, the complexity of the biological particle properties needed to remain simple to keep computational cost manageable and due to a lack of high resolution validation data.

Our model design does not resolve more complex ecosystem dynamics such as nutrient limitation and grazing by higher trophic levels. 
The Lagrangian model is performed offline, meaning it is not coupled to the Eulerian model that calculates the hydrodynamics and is performed after the fact.
Therefore, modeling the advection and dispersal of changes in concentration fields e.g. nutrients due to growth or remineralization was not easily possible.
Future modeling efforts could couple the Lagrangian model to a Eulerian model that disperses changes in concentrations fields by biotic activity throughout the model domain.
However, this would have drastically increased both developing and computational time to a point where it would have been infeasible in our time frame and due to the lack of appropriate validation data.
The key draw back of this is that growth rates could only be modeled as a constant rate in the current model description, similar to "ad libitum" experiments. 
This can lead to systematic errors in estimating population growth.
In reality, it can happen that nutrient limitation, which slows down the growth of the population, can occur, especially in the most light-saturated areas near the shore. For this reason, we may overestimate the role of shallow areas in our model.


To be consistent with the complexity of the representation of biotic mechanisms, we use a simplistic light limitation.
Particles are expected to be light limited below a water depth of 1 \unit{m} and not light limited above this threshold. 
We have not included a more complex light limitation model that takes into account current light availability and attenuation.
A more realistic formulation of light limitation could particularly favor particles that exhibit diel vertical migration.
\texttt{I tried to avoid getting into the "light availability" and "attenuation" topic. However, I feel writing it all reviewers will not be to happy with the current "bright during night" implementation. Do you feel I might get away with it? Or should I try to implement it before I submit it because they will ask for it anyway?}
A process we mostly ignore in our study is dormancy.
Our organisms can survive for $28$ days in light limited waters.
However, phytoplankton species have life stages in which they can remain dormant for a longer period of time and germinate again when they find themselves in more favorable waters \citep{ThomasAnderson1998}.
In the process of choosing the light limitation threshold, we conducted sensitivity studies testing the effect of higher light budgets.
We found that light budgets over 3 months begin to significantly increase the survivability of sinking organisms, when we crudely assume that they could still reproduce under these conditions.
Whether dormancy plays a significant role in an environment where the river bed is continuously dredged is unknown.

Another limitation in our modelling efforts is the lack of sub-grid-resolution structure on the shores.
In our representation we assume perfectly flat surfaces with a median distance between nodes of approximately 60 \unit{m}. 
This 'polished' model representation can lead to an underestimation of the retention success, since the surface area on which phytoplankton organisms can settle is underestimated.
In reality, vegetation, rocks or other surface irregularities cause a larger surface area on which the phytoplankton organisms can settle.

While our model does have a sedimentation and resuspension mechanic based on critical sheer velocities we still assume a static bathymetry with sediments not able to move or bury particles. 
This masks potential losses due to particles being buried but also decreases resuspension times.




Our results clearly suggest the importance of tidal flats and shallow areas along the river banks for the successful maintenance of primary production in the Elbe estuary. 
However, their effect can currently not be quantified due to the lack of validation data.
Chlorophyll data with a sufficient temporal and spacial resolution is only gathered in the center of the river.
Future monitoring efforts should therefore also include data along the river shores on tidal flats or shore-to-shore to quantify the effect of potential future changes by dredging, diking or restoration attempts.

For several decades, the annual average chlorophyll concentration in the Elbe estuary has been decreasing (data available at www.fgg-elbe.de/elbe-datenportal.html or see \citep{Hardenbicker2014,Schol2014}), while upstream concentrations do not show this effect.
The reasons for this are not fully understood, but one possible reason is the increase in dredge activity.
This increases the average turbidity and thus the aphotic depth, reducing the volume of water in which phytoplankton can grow.
A large fraction of the phytoplankton measured upstream of Hamburg harbor consists of diatoms \citep{Muylaert1999}, which typically have negative buoyancy \citep{Passow1991}, making them particularly susceptible to sinking in light-limited waters.
Our finding that sinking particles have a harder time surviving in the estuary supports this theory.

Another mechanism that might in part explain the drop in phytoplankton concentration at the bathymetric jump, which is not yet explored in our model, is the phytoplankton stickiness.
Phytoplankton, especially blooming one, has been shown to be sticky due to exudates \citep{Kiørboe1993,VanderLee2000,Dutz2005}. 
Some phytoplankton also produce transparent exopolymer particles, which increase their stickiness to other particles \citep{Windler,DeBrouwer2005}.
We suspect that this in combination with  higher turbidity induced by dredging results in losses due to plankton aggregates sticking to negatively buoyant suspended matter and subsequently sinking to the ground where they are starved of light.
A future model study could create estimates on chlorophyll concentration losses caused by this effect.


\conclusions  %% \conclusions[modified heading if necessary]
In this study, we investigated the role of different possible retention strategies for drifting phytoplankton organisms to remain in an estuarine environment.
As estuarine ecosystems are generally highly productive areas and of great value for anthropogenic use, maintaining their productivity is extremely important. However, whether and how drifting phytoplankton populations avoid being transported into unfavorable marine conditions has not been well understood. 
Our model simulations suggest that realistically high growth rates are sufficient for populations to prevent decline of estuarine populations.
Moreover, buoyancy and strong diel vertical migration support retention within the estuary.
The reason for this is that phytoplankton organisms with strong buoyancy or the potential to move upward in the water column are more likely to be transported to shallow areas near the river banks or tidal flats, where currents are slower and the chances of settlement under sufficiently light conditions to allow reproduction are higher than in deeper waters.
These results illustrate the importance of shallow nearshore areas for the productivity of estuarine ecosystems.
Our results also highlight the need for informed ecosystem-based management to avoid the degradation of estuarine ecosystems by dredging and diking activities.




%% The following commands are for the statements about the availability of data sets and/or software code corresponding to the manuscript.
%% It is strongly recommended to make use of these sections in case data sets and/or software code have been part of your research the article is based on.

% \codeavailability{TEXT} %% use this section when having only software code available


% \dataavailability{TEXT} %% use this section when having only data sets available


\codedataavailability{Input data can be requested from Johannes Pein (johannes.pein@hereon.de). OceanTracker's source code is available at\url{github.com/oceantracker/oceantracker}. Model configuration and output is available at \url{doi.org/10.25592/uhhfdm.13235}} %% use this section when having data sets and software code available


% \sampleavailability{TEXT} %% use this section when having geoscientific samples available


% \videosupplement{TEXT} %% use this section when having video supplements available


% \appendix

% \section{}    %% Appendix A

% \subsection{}     %% Appendix A1, A2, etc.

% \noappendix       %% use this to mark the end of the appendix section. Otherwise the figures might be numbered incorrectly (e.g. 10 instead of 1).

%% Regarding figures and tables in appendices, the following two options are possible depending on your general handling of figures and tables in the manuscript environment:

%% Option 1: If you sorted all figures and tables into the sections of the text, please also sort the appendix figures and appendix tables into the respective appendix sections.
%% They will be correctly named automatically.

%% Option 2: If you put all figures after the reference list, please insert appendix tables and figures after the normal tables and figures.
%% To rename them correctly to A1, A2, etc., please add the following commands in front of them:

% \appendixfigures  %% needs to be added in front of appendix figures

\appendixtables   %% needs to be added in front of appendix tables

%% Please add \clearpage between each table and/or figure. Further guidelines on figures and tables can be found below.



\authorcontribution{LS and RV contributed to conception of the study.
LS designed the studies details and organized the hydrodynamic data.
RV provided the source code for OceanTracker.
RS,LS improved on the original physical model and LS developed the biological model.
LS performed the model simulations, post-processing, and visualization.
LS wrote the draft of the manuscript.
All authors contributed to manuscript revision, read, and approved the submitted version.}



\competinginterests{The authors declare that they have no conflict of interest.} %% this section is mandatory even if you declare that no competing interests are present

% \disclaimer{TEXT} %% optional section

\begin{acknowledgements}
We thank Johannes Pein for providing and supporting the implementation of the hydrodynamic data.
Further, we thank Jana Hinners, Sina Remmers and Philipp Porada for providing helpful comments on the manuscript.
This study was funded by the Deutsche Forschungsgemeinschaft (DFG, German Research Foundation) within the Research Training Group 2530: “Biota-mediated effects on Carbon cycling in Estuaries” (project number 407270017), contribution to Universität Hamburg and Leibniz-Institut für Gewässerökologie und Binnenfischerei (IGB) im Forschungsverbund Berlin e.V.
\end{acknowledgements}




%% REFERENCES

%% The reference list is compiled as follows:

% \begin{thebibliography}{}
  
% \end{thebibliography}

%% Since the Copernicus LaTeX package includes the BibTeX style file copernicus.bst,
%% authors experienced with BibTeX only have to include the following two lines:
%%
\bibliographystyle{copernicus}
\bibliography{references.bib}
%%
%% URLs and DOIs can be entered in your BibTeX file as:
%%
%% URL = {http://www.xyz.org/~jones/idx_g.htm}
%% DOI = {10.5194/xyz}


%% LITERATURE CITATIONS
%%
%% command                        & example result
%% \citet{jones90}|               & Jones et al. (1990)
%% \citep{jones90}|               & (Jones et al., 1990)
%% \citep{jones90,jones93}|       & (Jones et al., 1990, 1993)
%% \citep[p.~32]{jones90}|        & (Jones et al., 1990, p.~32)
%% \citep[e.g.,][]{jones90}|      & (e.g., Jones et al., 1990)
%% \citep[e.g.,][p.~32]{jones90}| & (e.g., Jones et al., 1990, p.~32)
%% \citeauthor{jones90}|          & Jones et al.
%% \citeyear{jones90}|            & 1990



%% FIGURES

%% When figures and tables are placed at the end of the MS (article in one-column style), please add \clearpage
%% between bibliography and first table and/or figure as well as between each table and/or figure.

% The figure files should be labelled correctly with Arabic numerals (e.g. fig01.jpg, fig02.png).


%% ONE-COLUMN FIGURES

%%f
%\begin{figure}[t]
%\includegraphics[width=8.3cm]{FILE NAME}
%\caption{TEXT}
%\end{figure}
%
%%% TWO-COLUMN FIGURES
%
%%f
%\begin{figure*}[t]
%\includegraphics[width=12cm]{FILE NAME}
%\caption{TEXT}
%\end{figure*}
%
%
%%% TABLES
%%%
%%% The different columns must be seperated with a & command and should
%%% end with \\ to identify the column brake.
%
%%% ONE-COLUMN TABLE
%
%%t
%\begin{table}[t]
%\caption{TEXT}
%\begin{tabular}{column = lcr}
%\tophline
%
%\middlehline
%
%\bottomhline
%\end{tabular}
%\belowtable{} % Table Footnotes
%\end{table}
%
%%% TWO-COLUMN TABLE
%
%%t
%\begin{table*}[t]
%\caption{TEXT}
%\begin{tabular}{column = lcr}
%\tophline
%
%\middlehline
%
%\bottomhline
%\end{tabular}
%\belowtable{} % Table Footnotes
%\end{table*}
%
%%% LANDSCAPE TABLE
%
%%t
%\begin{sidewaystable*}[t]
%\caption{TEXT}
%\begin{tabular}{column = lcr}
%\tophline
%
%\middlehline
%
%\bottomhline
%\end{tabular}
%\belowtable{} % Table Footnotes
%\end{sidewaystable*}
%
%
%%% MATHEMATICAL EXPRESSIONS
%
%%% All papers typeset by Copernicus Publications follow the math typesetting regulations
%%% given by the IUPAC Green Book (IUPAC: Quantities, Units and Symbols in Physical Chemistry,
%%% 2nd Edn., Blackwell Science, available at: http://old.iupac.org/publications/books/gbook/green_book_2ed.pdf, 1993).
%%%
%%% Physical quantities/variables are typeset in italic font (t for time, T for Temperature)
%%% Indices which are not defined are typeset in italic font (x, y, z, a, b, c)
%%% Items/objects which are defined are typeset in roman font (Car A, Car B)
%%% Descriptions/specifications which are defined by itself are typeset in roman font (abs, rel, ref, tot, net, ice)
%%% Abbreviations from 2 letters are typeset in roman font (RH, LAI)
%%% Vectors are identified in bold italic font using \vec{x}
%%% Matrices are identified in bold roman font
%%% Multiplication signs are typeset using the LaTeX commands \times (for vector products, grids, and exponential notations) or \cdot
%%% The character * should not be applied as mutliplication sign
%
%
%%% EQUATIONS
%
%%% Single-row equation
%
%\begin{equation}
%
%\end{equation}
%
%%% PHYSICAL UNITS
%%%
%%% Please use \unit{} and apply the exponential notation


\end{document}
